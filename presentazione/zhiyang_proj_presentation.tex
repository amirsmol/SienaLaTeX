%	This is written by Zhiyang Ong for his presentation about his term project for the class, ``ECEN 751 Advanced Computational Methods for Integrated System Design,'' in Spring 2014 at Texas A&M University.



%%%%%%%%%%%%%%%%%%%%%%%%%%%%%%%%%%%%%%%%%%%%%%
%	Preamble

%	Use the Beamer package to create the presentation slides.
\documentclass[xcolor={usenames,dvipsnames},hyperref={hyperindex,bookmarks}]{beamer}
%	Background color: Set it to blue.
%\setbeamercolor{background canvas}{bg=blue}
%	\setbeamercolor{normal text}{bg=white,fg=yellow}
%\setbeamercolor{normal text}{fg=white}
%	\setbeamercolor{title}{fg=yellow,bg=white}
%\setbeamercolor{title}{fg=yellow}
%	\setbeamercolor{titlelike}{fg=yellow,bg=white}
%\setbeamercolor{block title alerted}{fg=white,bg=yellow}


%%%%%%%%%%%%%%%%%%%%%%%%%%%%%%%%%%%%%%%%%%%%%%
%	Table of Contents
%	\AtBeginSection[]
%	{
%		\begin{frame}
%			\frametitle{\textcolor{yellow}{Table of Contents}}
%			\textcolor{yellow}{\tableofcontents[currentsection]}
%	%		\tableofcontents[currentsection,currentsubsection]
%		\end{frame}
%	}



%%%%%%%%%%%%%%%%%%%%%%%%%%%%%%%%%%%%%%%%%%%%%%
%	Main document
\begin{document}


%%%%%%%%%%%%%%%%%%%%%%%%%%%%%%%%%%%%%%%%%%%%%%
%	Slide 1

%	Set the background picture.
{
%\usebackgroundtemplate{\includegraphics[width=\paperwidth,height=\paperheight,keepaspectratio]{./pics/cover}}

	\title[Cover] % (optional, only for long titles)
%	{Optimization Framework for Cancer Radiation Therapy}
	{Pattern Recognition-based Reconstruction \\of Neural Networks}
	\subtitle{From Fluorescence Imaging of Neural Activity}
	\author[Author, Anders] % (optional, for multiple authors)
	{Zhiyang Ong}%	\inst{a}}
	\institute[TAMU] % (optional)
	{
		%\inst{a}
		Department of Electrical and Computer Engineering \\
		Dwight Look College of Engineering \\
		Texas A\&M University \\
		College Station, TX
	}
	\date{\today} % (optional)
	%\subject{Discrete Optimization}
	\frame{\titlepage}
%	For setting the background picture.
}

%	AUTHOR, ``TITLE,'' in {\it The New York Times: The Opinion Pages: Op-Ed Contributor}, The New York Times Company, New York, NY, MONTH DATE, YEAR.

%%%%%%%%%%%%%%%%%%%%%%%%%%%%%%%%%%%%%%%%%%%%%%
%	Acknowledgments

%	\begin{frame}
%		\frametitle{\textcolor{yellow}{Acknowledgments}}
%		Dr. Sani Nassif, Research Staff Member at IBM Research \\
%		\ \\
%		Prateek Tandon, Ph.D. student at Carnegie Mellon University's Robotics Institute \\
%		\ \\
%		Dr. Alexander Mitev, embedded systems engineer
%	\end{frame}




%\section{Test section one}

%	  \begin{frame}
%	    %\frametitle{\textcolor{yellow}{This is the second slide}}
%	    \frametitle{This is the second slide}
%	    %\framesubtitle{\textcolor{Green}{A bit more information about this}}
%	    \framesubtitle{A bit more information about this}
%	    %More content goes here
%	    Greetings \cite[slide 9]{Ganesh2013}
%	    \textcolor{white}{Hello World!}
%	  \end{frame}



%%%%%%%%%%%%%%%%%%%%%%%%%%%%%%%%%%%%%%%%%%%%%%
%	Slide 2
\begin{frame}
	\frametitle{Background Information \& Existing Approaches}
	\framesubtitle{Techniques for Pattern Recognition of Structural Properties of the Brain}
	
	\begin{columns}[t] % contents are top vertically aligned
		\begin{column}[T]{5cm} % each column can also be its own environment
		
		Problem: Inadequate knowledge of the structure of the human connectome, which can facilitate medical diagnosis and treatment. \\
		\ \\
		Objective: Determine connectivity of neural network via pattern recognition. \\
		\ \\
		Existing solutions: statistical causal inference, score-based methods
		\end{column}
		
		\begin{column}[T]{5cm} % alternative top-align that's better for graphics
			\begin{figure}
			\centering
			\includegraphics[height=2.8in]{./pics/data_pdtn_wkflow2}
%			\caption{Workflow for wetware- and software-/challenge- based neural network reconstruction: (a) workflow used to construct the neural network using wetware \cite{ChaLearn2014a,Stetter2012}}
			\end{figure}
		\end{column}
	\end{columns}
\end{frame}



%%%%%%%%%%%%%%%%%%%%%%%%%%%%%%%%%%%%%%%%%%%%%%
%	Slide 3
\begin{frame}
	\frametitle{Proposed Approach}
	\framesubtitle{Description of attempted solutions.}
	
	Use the workflow from the Kaggle competition \cite{ChaLearn2014a,Stetter2012} to determine the connectivity of a neuron $n_{i}$ and other neurons $n_{j}, \forall j \in N = \{n_{1}, n_{2}, \dots, n_{m}\}$ that is connected to its dendrites (inputs) and axons (outputs). \\
	\ \\
	Focus on pattern recognition via semidefinite programming-based support vector machine\dots\ Battle with the No Free Lunch theorem. \\
	\ \\
	Modified focus on observed time series, with a focus on techniques of statistical causal inference that can capture spatio-temporal dynamics.
\end{frame}


%%%%%%%%%%%%%%%%%%%%%%%%%%%%%%%%%%%%%%%%%%%%%%
%	Slide 4
\begin{frame}
	\frametitle{Experimental Results \& Conclusions}
	\framesubtitle{Analysis of Experimental Data and Conclusions}
	
	\begin{figure}
		\centering
		\includegraphics[height=2.6in]{./pics/work_in_progress2}
%		\caption{Workflow for wetware- and software-/challenge- based neural network reconstruction: (a) workflow used to construct the neural network using wetware \cite{ChaLearn2014a,Stetter2012}}
	\end{figure}
\end{frame}




%%%%%%%%%%%%%%%%%%%%%%%%%%%%%%%%%%%%%%%%%%%%%
%	Slide 5
%	References
{\linespread{1}
\begin{frame}
	\frametitle{References}
	\bibliographystyle{plain}
	\bibliography{/data/research/antipastobibtex/references}
\end{frame}
}
%%%%%%%%%%%%%%%%%%%%%%%%%%%%%%%%%%%%%%%%%%%%%
\end{document}