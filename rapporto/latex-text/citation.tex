%	This is written by Zhiyang Ong as a template for writing text in LaTeX.

%	The MIT License (MIT)

%	Copyright (c) <2014> <Zhiyang Ong>

%	Permission is hereby granted, free of charge, to any person obtaining a copy of this software and associated documentation files (the "Software"), to deal in the Software without restriction, including without limitation the rights to use, copy, modify, merge, publish, distribute, sublicense, and/or sell copies of the Software, and to permit persons to whom the Software is furnished to do so, subject to the following conditions:

%	The above copyright notice and this permission notice shall be included in all copies or substantial portions of the Software.

%	THE SOFTWARE IS PROVIDED "AS IS", WITHOUT WARRANTY OF ANY KIND, EXPRESS OR IMPLIED, INCLUDING BUT NOT LIMITED TO THE WARRANTIES OF MERCHANTABILITY, FITNESS FOR A PARTICULAR PURPOSE AND NONINFRINGEMENT. IN NO EVENT SHALL THE AUTHORS OR COPYRIGHT HOLDERS BE LIABLE FOR ANY CLAIM, DAMAGES OR OTHER LIABILITY, WHETHER IN AN ACTION OF CONTRACT, TORT OR OTHERWISE, ARISING FROM, OUT OF OR IN CONNECTION WITH THE SOFTWARE OR THE USE OR OTHER DEALINGS IN THE SOFTWARE.

%	Email address: echo "cukj -wb- 23wU4X5M589 TROJANS cqkH wiuz2y 0f Mw Stanford" | awk '{ sub("23wU4X5M589","F.d_c_b. ") sub("Stanford","d0mA1n"); print $5, $2, $8; for (i=1; i<=1; i++) print "6\b"; print $9, $7, $6 }' | sed y/kqcbuHwM62z/gnotrzadqmC/ | tr 'q' ' ' | tr -d [:cntrl:] | tr -d 'ir' | tr y "\n"

%%%%%%%%%%%%%%%%%%%%%%%%%%%%%%%%%%%%%%%%%%%%%%



%%%%%%%%%%%%%%%%%%%%%%%%%%%%%%%%%%%%%%%%%%%
\section{Referencing Information}
\label{sec:RefInfo}

Here is how I can reference common resources: \vspace{-0.3cm}
\begin{enumerate} \itemsep -4pt
\item For online resources: \vspace{-0.3cm}
	\begin{enumerate} \itemsep -2pt
	\item Author, ``Title of web page,'' in {\it Title of Primary Web Site}, Name of Publisher/Organization/Individual, Address, Month Date, Year. Available online at: \url{http://www.webpage.url/}; last accessed on June 2, 2014.	% Available online at: \url{URL}; last accessed on June 2, 2014.
	\item Regarding entries for my {\sc Bib}\TeX\ database, insert the following to the ``howpublished'' field: Available online at: \url{http://www.webpage.url/}; June 11, 2012 was the last accessed date.	% Available online at: \url{URL}; June 11, 2012 was the last accessed date
	\end{enumerate}
\item DOI field in {\sc Bib}\TeX\ should be indicated as a URL: \url{http://dx.doi.org/DOI}.	% http://dx.doi.org/DOI
\item To enter a summary of a paper that I have written into a report, enter it as a section (or subsection or subsubsection) with the following ``fields'': \vspace{-0.3cm}
	\begin{enumerate} \itemsep -2pt
	\item In the title of the section, indicate the title of the paper and its abbreviation (i.e., its {\sc Bib}\TeX\ key).
	\item Terse summary: Summary of the paper in 2-3 lines.
	\item Not-so-concise summary and highlights. Summarize the publication in $\leq$ 2 pages. For publications that are not survey papers nor literature review, highlight the advantages and disadvantages of the described techniques/innovations. For survey papers nor literature review publications, summarize the primary publications that was mentioned in the survey/review.
	\item Other notes about the publication: Insert important figures and equations, among other details about the paper.
	\end{enumerate}
\item {\it BibDesk} only creates a folder for publications with non-empty author fields. Hence, when entering a {\sc Bib}\TeX\ into my {\sc Bib}\TeX\ database, enter the names of the editors into the {\tt author} field. When citing the publication, use a script to shift the content of the {\tt author} field into the {\tt editor} field.

 When citing from this entry, shift the names of the authors from the author field to the editor field.
\item 
\item 
\item 
\item 
\item 
\item 
\item 
\item 
\item 
\item 
\item 
\item 
\item 
\item 
\item 
\end{enumerate}